\documentclass[12pt]{article}
\usepackage{graphicx}
\usepackage[british,UKenglish,USenglish,english,american]{babel}
\usepackage{tocbibind}
\usepackage[toc,page]{appendix}

%--------------dodatkowe pakiety:-----------
\usepackage{mathtools}
\usepackage{amsfonts}
\usepackage{amsmath}
\usepackage{amsthm}
\usepackage{array}

%-------------------links------------------
\usepackage{hyperref}
\hypersetup{
    colorlinks=true,
    linkcolor=blue,
    filecolor=magenta,      
    urlcolor=cyan,
}
 
\urlstyle{same}
%------placing image caption at the bottom---------
\usepackage{floatrow}
%--------------allows pic alligning----------------
\usepackage[export]{adjustbox}
%----------------for console output-----------------
\usepackage{listings}

\begin{document}

\title{Introduction to CUDA and OpenCL}
\author{Ilona Tomkowicz, Zofia Pieńkowska}

\maketitle
\tableofcontents
\newpage
%--------------------------------------------------------------------

\section{Data structure limits} 
The biggest data structure that could be used in sample vector add project was 2 pow 27.
\noindent Console output:
\begin{lstlisting}[language=bash]
  $ ./executable
  [Vector addition of 268435456 elements]
  Copy input data from the host memory to the CUDA device
  CUDA kernel launch with 262144 blocks of 1024 threads
  Copy output data from the CUDA device to the host memory
  Test PASSED
  Done
\end{lstlisting}



\subsection{How large data are handled successfully} 
\subsection{What is going on in this experiment}
\subsection{When does the code gives errors?}
\section{Optimal grid layout search} 
\subsection{Layout experiments}
\subsection{Conclusions}
\end{document}